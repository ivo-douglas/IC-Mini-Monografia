%%%%%%%%%%%%%%%%%%%%%%%%%%%%%%%%%%%%%%%%%%%%%%%%%%%%%%%%%%%%%%%%%%%%%%
% How to use writeLaTeX: 
%
% You edit the source code here on the left, and the preview on the
% right shows you the result within a few seconds.
%
% Bookmark this page and share the URL with your co-authors. They can
% edit at the same time!
%
% You can upload figures, bibliographies, custom classes and
% styles using the files menu.
%
%%%%%%%%%%%%%%%%%%%%%%%%%%%%%%%%%%%%%%%%%%%%%%%%%%%%%%%%%%%%%%%%%%%%%%

\documentclass[12pt]{article}

\usepackage{sbc-template}

\usepackage{graphicx,url}

\usepackage[brazil]{babel}   
\usepackage[utf8]{inputenc}  


\begin{document} 
     
\sloppy

\title{Caixa Tem e Você Não}

\author{Danilo dos Santos de Oliveira, Douglas Ivo Martins de Moraes}

\maketitle



\begin{abstract}
  This meta-paper describes the style to be used in articles and short papers
  for SBC conferences. For papers in English, you should add just an abstract
  while for the papers in Portuguese, we also ask for an abstract in
  Portuguese (``resumo''). In both cases, abstracts should not have more than
  10 lines and must be in the first page of the paper.
\end{abstract}
     
\begin{resumo}
  Este meta-artigo descreve o estilo a ser usado na confecção de artigos e
  resumos de artigos para publicação nos anais das conferências organizadas
  pela SBC. É solicitada a escrita de resumo e abstract apenas para os artigos
  escritos em português. Artigos em inglês deverão apresentar apenas abstract.
  Nos dois casos, o autor deve tomar cuidado para que o resumo (e o abstract)
  não ultrapassem 10 linhas cada, sendo que ambos devem estar na primeira
  página do artigo.
\end{resumo}


\section{Contextualizar e definir o problema atacado pelo produto e/ou tecnologia de informação alvo} \label{sec:firstpage}

	No atual estado da pandemia que agrava o mundo várias crises e problemas estão surgindo, dentre eles se encontra o desemprego. Sendo assim, como uma forma de o governo solucionar esse problema foi criado o auxílo emergencial e a Caixa Econônimica Federal ficou responsável pela iniciativa, o intuito é conceder o valor de R\$ 600 somente para aqueles que sofrem o risco de serem afetados pelo desemprego ou ainda pela deficiência econômica causada pela pandêmia.
\subsection{Regras do auxílio emergêncial}	
Com isso, o auxílio emergência visa contemplar somente aqueles com:
\linebreak
\linebreak
*idade; Maior de 18 anos (exceto mães)
\linebreak
\linebreak
*Ocupação; Trabalhador sem carteira assinada, autônomo, MEI (microempreendedor individual), desempregado, contribuinte individual da Previdência 
\linebreak
\linebreak
*Renda; Renda por pessoa da família de até R\$ 522,50 ou renda familiar de até R\$ 3.135. E não ter recebido rendimentos tributáveis acima de R\$ 28.559,70 em 2018
\linebreak
\linebreak
E ainda não tem direito quem já recebe seguro-desemprego, BPC, aposentadoria ou pensão
\linebreak
\linebreak
\subsection{Problemas Solucionados}
	No entanto, este programa trouxe outra adversidade, que é a superlotação e grandes filas em agências para o saque ou ainda para o cadastramento para a análise do auxílio emergêncial. Com isso, em 23 de maio, a caixa lançou o aplicativo, Caixa | Auxílio Emergêncial, para que esses problemas possam ser resolvídos.

\subsection{Detalhes do aplicativo}
	Portanto, este é um aplicativo que compreende o négocio do tipo G2C, propondo um contato mais direto e imediato entre o governo e às pessoas que buscam pelo auxílio emergencial, exigindo o acesso a uma conexão de rede internet para saques e cadastramentos.
\subsection{Problemas causados}
	Embora as filas tenham sidas reduzidas ao mínimo, problemas de cunho ético foram surgindo. Trata-se de pessoas burlando o sistema para receber o auxílio, isto é, mesmo estas que não fazem parte do grupo que é contemplado pelo programa. Como foi o caso do filho de William Bonner, Vinícius Bonner, que teve sua identidade roubada e usada para um cadastramento do auxílio emergêncial e mesmo assim, teve o status aprovado.
\linebreak
\linebreak
	Então percebemosos que, o sistema não é muito confiável na analise dos dados, pois de acordo com William Bonner, as condições sócio-econômicas de seu filho não atendem aos critérios do grupo de necessitados do auxílio emergêncial, portanto ele não tem direito aos 600 reais, porém ainda obteve o status de aprovado. Logo pode-se concluir que não há eficiência na detecção de fraudes, onde um cidadão pode-se passar por outro, a fim de se beneficiar ou ainda abusar do sistema do auxílio emergêncial. 
\section{Apresentar os objetivos da monografia}
	Esta monografia tem o objetivo de auxiliar e dar opções para contornar as dificuldades que o Auxílio Emergencial esta sofrendo em relação à análise de dados de seu programa atualmente, visando uma possibilidade de solucionar sua deficiência na manipulação de dados, por meio de um gerenciamento de banco de dados avançado que beneficie, viávelmente, o sistema de análise de informações e apoio à decisão sem danificar informações pessoais.

\section{Apresentar a solução de TI em alto nível e também relacionando-a com as áreas de SI (Banco de Dados, infraestrutura de TI, Engenharia de Software, Análise de Informações e Apoio à Decisão, Algoritmos, Projetos, Segurança, Governança, Uso da Internet …);}
Conteúdo
\linebreak
\linebreak
Conteúdo

\section{Apresentar as conclusões}
Conteúdo

\section{References}

Bibliographic references must be unambiguous and uniform.  We recommend giving
the author names references in brackets, e.g. \cite{knuth:84},
\cite{boulic:91}, and \cite{smith:99}.

The references must be listed using 12 point font size, with 6 points of space
before each reference. The first line of each reference should not be
indented, while the subsequent should be indented by 0.5 cm.

\bibliographystyle{sbc}
\bibliography{sbc-template}

\end{document}
